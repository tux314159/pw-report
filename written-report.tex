\documentclass[a4paper]{article}

\usepackage[]{graphicx}
\usepackage[style=apa]{biblatex}
\usepackage[colorlinks=true,citecolor=blue,linkcolor=blue]{hyperref}
\usepackage[]{microtype}
\usepackage[table]{xcolor}
\usepackage[]{multicol}
\usepackage[toc,page]{appendix}
\usepackage[capitalise]{cleveref}
\usepackage[]{float}
\usepackage[margin=1.3in]{geometry}
\usepackage[]{setspace}
\usepackage[]{latexsym}
\usepackage[]{ragged2e}

\graphicspath{{./graphs/}{./images/}}
\addbibresource{written-report.bib}

\begin{document}
\begin{titlepage}
  \Centering
  \Large{Raffles Institution \\ Year 5 Project Work \\ Written report} \\
  \includegraphics[scale=0.5]{ri-school-crest.png} \\
    \huge{The problem of youths falling for online scams, and ways to prevent this.} \\
  \vspace{0.5cm}
  \small{\emph{Word Count: ...}} \\
  \vspace{0.5cm}
  \large{
    \textbf{Authors}: \\
    Alicia Chan \\
    Vera Tay\footnote{Group leader} \\
    Isaac Yeo \\
    Lyu Junwei \\
    Karthik \\
    \vspace{1cm}
    \begin{tabular}{r@{:}l}
      \textbf{Class} & \hspace{1cm} 24S02C \\
    \end{tabular}

  }
\end{titlepage}

\newpage

\pagenumbering{Roman}

\begin{abstract}
  \addcontentsline{toc}{section}{Abstract}
  \noindent
  ...?
\end{abstract}

\newpage


\tableofcontents

\newpage

\pagenumbering{arabic}

\begin{multicols}{2}

    \section{Identification of problems}
    \subsection{Choice of topic}
    \paragraph{} We aim to reduce the number of young adults falling for scams.
    \subsection{Rationale for choice of topic}
    \paragraph{} As society becomes increasingly digitalised,
    many day-to-day processes are shifting online, from
    e-commerce \parencite{ITA.2022} to banking services
    \parencite{Statista.2023}. Consequently, young adults are spending
    more time on the Internet, becoming vulnerable to and falling for
    online scams.

    \subsection{Significance of problem}
    \paragraph{} The amount of money lost to scams has
    increased from \$632 million in 2021 to \$660 million in 2022
    \parencite{Chua.2023}. To prevent further losses, it is imperative
    that we reduce the number of people falling for scams, starting with
    the Target Group, who is relatively more vulnerable.

    \subsection{Rationale for Choice of Target Group}
    \paragraph{} More than 26\% of scam victims are between 20 and 29
    years old \parencite{Chua.2023}, which is relatively high compared
    to other age groups, e.g. those aged 60-69 only constituted 6.5\%
    of scam victims. Considering that this group will shortly be
    the main contributors to Singapore’s economy, if the problem is
    unaddressed or addressed insufficiently, Singapore will suffer even
    more unnecessary financial losses in the near future.

    \section{Analysis of Target Group and Problem}
    \subsection{Casual factors}
    \subsubsection{Complacent attitudes}
    \paragraph{} The Target Group tends to have the misconception that
    they are immune to scams because they are more tech-savvy compared
    to older generations, adopting an “I know, but I don’t really
    care” attitude \parencite{BrandStudio.2020}. As a result, they
    become overconfident when navigating the Internet, believing that
    since they do not fit the perceived elderly profile of a scam victim,
    they are unlikely to fall for scams. Thus, they are less careful
    online \parencite{Carlson.2022}, resulting in them easily falling
    prey to scams.
    \subsubsection{Lack of knowledge}
    \paragraph{} Due to their complacency, the Target Group is less
    willing to learn more about identifying and preventing scams. For
    instance, investment and job scams were the top 2 scams in 2022
    \parencite{Chua.2023}; as the Target Group has just entered the
    workforce, there is understandably a heightened want and need for
    money, making the scams seem like attractive propositions. However,
    they tend to act impulsively without sufficient knowledge on the
    situation, driven by their greed for what seems like a ``good deal''.

    \subsection{Current measures and gaps}
    \paragraph{} The National Crime Prevention Council (NCPC) has launched an
	anti-scam campaign with the 2023 tagline ``I can ACT against
    scams'' \parencite{Sun.2023}. The NCPC has also launched
	a website called ScamAlert ("Scam Alert," n.d.). Key components
	of ScamAlert include more details on NCPC’s ACT strategy,
	information on common scams, true stories of people getting
	scammed and links to news articles regarding scams.



\end{multicols}

\newpage

\nocite{*}
\printbibliography[heading=bibintoc,title={References}]

\newpage

%TC:ignore

\begin{appendices}
\end{appendices}


%TC:endignore
\end{document}
